\section{Sediment disaggregation}

\subsection{Silt to sand}

Sieve cleaning and storage: 
Thoroughly rinse sieves after each use to remove excess stuck sediments 
To rinse turn sieve over and run water over under side to remove particles without forcing them through the sieve.
When rinsed dry the sieves to remove most of the water 
Use ethanol or isopropyl alcohol in squeeze bottle and spray over bottom of sieves 
Place cleaned sieves in oven to dry overnight 
Store in dry area 

Preparing Sediments for Sieving: 
Find clay sediment samples stored in buckets in lab refrigerator and remove a handful sized chunk of clay 
Place clay in large glass beaker and add water to ~ midway in the beaker 
Break up large clumps of clay by hand and stir with long glass stirbars 
Once the clay is fully broken up and there are no large clumps in the beaker, the sediment mixture is ready for sieving 

Wet Sieving: 
Place wet sieves in the sink 
Ensure that sieves are aligned with the 200mm sieve on top and the large 75\unit{\micro\meter}m sieve on the bottom.
Carefully and slowly pour clay slurry from the beaker onto the top 200mm sieve
After the clay/water mixture is poured, use a plastic hose with low water flow to rinse the top sieve until it is clear of fine particles and only large grains remain
Remove 200mm sieve and place it aside 
Often fine clay particles clog the 75\unit{\micro\meter}m sieve so it may be necessary to move water through the sieve by disrupting the water  
To clear clear clay you can also use water (very gently through the hose) to move the clay through the sieve
*Note* if the 75\unit{\micro\meter}m sieve is clogged do not try to force sediments through the sieve and when using hand or stir bar to clear sieve do not put too much pressure on or scratch the sieve 

Once water is drained from the 75\unit{\micro\meter}m sieve, rinse several times with the hose and gentle water flow to removethe last of the clay so that only sandy particles remain. 
(The sandy material will generally look like white/black particles, while the clay will generally be grey or red)
Prep a clean empty beaker 
Use the hose to flush the sandy particles from the bottom of 75\unit{\micro\meter}m sieve into the beaker 
*Note* Be careful when rinsing the 75\unit{\micro\meter}m sieve we do not want to lose any of this material 
When you rinse sediments into the beaker, you will have a mix of sandy sediments and water in the beaker. Allow this to settle and pour off as much of the clean water as possible without losing any of the sediments (you just want most of the water out) 
Continue sieving and collecting sediments in the beaker 
*Note* be sure that you are only sieving the sediments from one site, and do not use all of the sediments from a single site set aside 200-500 grams of sediment from each site in bag or storage container

Drying Sediments: 

The beaker of collected, sieved, sandy sediment that has been decanted and prepped with Isopropanol can now be dried. 
Place the glass beaker in the oven at ~ 60\unit{\degreeCelsius} allow to dry for several hours 
Move dried sediments to a bag or container properly labeled with the sediments collection site name.  


\subsection{Clay}

\section{Sieving}