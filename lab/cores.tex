
\section{Splitting and cleaning coers}
Coming soon\ldots


\section{Core imaging}
Core imaging is accomplished by taking sequential digital photographs along the length of a core and stitching the individual images into a single continuous core scan.


\subsection{Image collection}
After cores are split and the surface is cleaned and prepared for imaging, place the split core in the core imaging box, aligning the bottom edge of the core with 0~cm on the scale. 
Replace the box cover and slide it to a position roughly over the midpoint of the core.

Remove the lens cap and turn on the camera, ensuring the core and scale are fully within the field of view. The camera shutter speed (1/50?\footnote{Graham needs to double check this so do not change if it is incorrect})and ISO (80) are set to optimize clear images with appropriate contrast for the lighting array. \textbf{Do not adjust these settings.}

Image both split halves of a core, labelling the respective halves `a' and `b'. 
\begin{enumerate}
    \item With the camera positioned over roughly the midpoint of the core. Press the Autofocus button (``AF~ON'') and allow the camera to focus on the core surface until a green circle appears in the lefthand corner of the display. Collect an image and visually confirm its quality. Delete this image and begin imaging the core.
    \item Collect the first image at the upper extent of the core.
    \item Collect images---progressing downcore---at 10~cm increments, using the subtle resting points of the imaging box as you slide the top along its track. Ensure that sequential images overlap by $\ge$1~cm.
    \item Repeat this process for subsequent core-halves, keeping careful track of the image sequence corresponding to each core-half. 
\end{enumerate}

Once you have completed a session of image collection, transfer the RAW image files to a computer. Distribute images to directories corresponding to individual core-halves, using the following directory hierarchy, for example for images from a core-half `a'.

\texttt{ResarchProjectFolder > core-scans > CoreName > a > \{images\}}


\subsection{Image processing}

\subsubsection{Optics corrections}

\subsubsection{Stitching the core scan}

